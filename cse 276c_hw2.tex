\documentclass[10pt,letterpaper]{article}

%XXX: Packages
\usepackage[margin=1in,headheight=110pt]{geometry} % corrects the margins
\usepackage[table,usenames,dvipsnames]{xcolor}      % color
\usepackage{extarrows}                              % http://ctan.org/pkg
\usepackage[shortlabels]{enumitem}
\usepackage{amsmath,amssymb,amsfonts,amsthm,dsfont} % math
\usepackage{algorithm,algorithmicx,listings}        % algorithms
\usepackage[noend]{algpseudocode}			        % necessary for algorithmicx
\usepackage{graphicx}
\usepackage{mathtools}                              % short insert text

\usepackage{subcaption}
\usepackage{tikz}      % plot finite state machine for MDP transition graph
\usetikzlibrary{automata, positioning, arrows.meta} % tikz package dependency

\usepackage[breaklinks=true, colorlinks, bookmarks=true, citecolor=Black, urlcolor=Violet,linkcolor=Black]{hyperref}

% XXX: Commands:
\def\argmin{\mathop{\arg\min}\limits}	%
\def\argmax{\mathop{\arg\max}\limits}	%
\newcommand{\indicator}{\mathds{1}}
\newcommand{\ceil}[1]{\lceil#1\rceil}
\newcommand{\floor}[1]{\lfloor#1\rfloor}
\DeclareMathOperator{\tr}{tr}
\DeclareMathOperator{\Var}{Var}
\newcommand{\txbx}[1]{\boxed{\text{#1}}}
\newcommand{\scaleMathLine}[2][1]{\resizebox{#1\linewidth}{!}{$\displaystyle{#2}$}}
\newcommand{\prl}[1]{\left(#1\right)}
\newcommand{\brl}[1]{\left[#1\right]}
\newcommand{\crl}[1]{\left\{#1\right\}}
\renewcommand{\P}{\mathbb{P}}
\newcommand{\F}{\mathcal{F}}
\newcommand{\TODO}[1]{{\color{red}#1}}

% XXX: Environments:
\newtheorem{theorem}{Theorem}
\newtheorem{proposition}{Proposition}
\newtheorem{corollary}{Corollary}
\newtheorem{lemma}{Lemma}
\theoremstyle{definition}
\newtheorem{definition}{Definition}
\newtheorem{assumption}{Assumption}
\newtheorem*{assumption*}{Assumption}
\newtheorem*{problem*}{Problem}
\newtheorem{problem}{Problem}
\theoremstyle{remark}
\newtheorem{remark}{Remark}
\newtheorem*{solution*}{Solution}

%%%%%%%%%%%%%%%%%%%%%%%%%%%%%%%%%%%%%%%%%%%%%%%
% \input{../common/sym.tex}
%%%%%%%%%%%%%%%%%%%%%%%%%%%%%%%%%%%%%%%%%%%%%%%

\def\thetitle{\textcolor{black}{Homework 2 Solutions}}
\def\theauthor{Shrey Kansal}

\hypersetup{
  pdfauthor={\theauthor},%
  pdftitle={\thetitle},%
  pdfsubject={CSE 276C HW2 Solutions}
}

\begin{document}

\section*{\thetitle}

\subsection*{Problems}
\begin{enumerate}[leftmargin=*,itemsep=9ex]
  \item Problem 1: \\
  Hello
  \item Problem 2:\\
  Since, we want 6 digit accuracy for the $cos(x)$ function, we say that our maximum error should be $10^{-6}$.\\
  The error is expressed by the following formula:
  \begin{equation}
    \mathbb{E} \leq \frac{1}{(n+1)!} \max_x{f^{n+1}(x)\prod_{i=0}^{n} (x - x_i)}, for\ x \in [0, \pi]
  \end{equation}
  where $n$ is the degree of polynomial used to interpolate the given function.

  \begin{enumerate}
    \item Linear Interpolation:
    For linear interpolation, $n=1$. Hence, using (1):
    \begin{equation}
      10^{-6} \leq \frac{1}{2!}\max_x{|\frac{d^3cos(x)}{dx^3}|}\max_x|\prod_{i=0}^{1} (x - x_i)|
    \end{equation}
    \[
      \max_x{|\frac{d^3cos(x)}{dx^3}|} = \max_x{|sin(x)|} = 1, for \ x \in [0, \pi]
    \]
    \\
    To get the max of product, we differentiate it with respect to $x$ as follows:
    \[
      \frac{d}{dx}(\prod_{i=0}^{1} (x - x_i)) = 0,
    \]
    \[
      \implies 2x - (x_0 + x_1) = 0
    \]
    \[
      \implies x = \frac{(x_0 + x_1)}{2}
    \]
    Substituting $x$ in (2):
    \[
      10^{-6} \leq \frac{1}{2!}\max|\frac{x_0 - x_1}{2}\frac{x_1 - x_0}{2}|
    \]
    Let $h = x_1 - x_0$ be the table spacing,
    \[
      10^{-6} \leq \frac{1}{2!}|\frac{-h^2}{4}|
    \]
    \[
      \implies h \geq \sqrt{8\times10^{-6}} = 0.0028
    \]
    Hence, the required table spacing is $h = 0.0028$.
    \\
    \item Quadratic Interpolation:
    For quadratic interpolation, $n=2$. Hence, using (1):
    \begin{equation}
      10^{-6} \leq \frac{1}{3!}\max_x{|\frac{d^4cos(x)}{dx^4}|}\max|\prod_{i=0}^{2} (x - x_i)|
    \end{equation}
    \[
      \max_x{|\frac{d^4cos(x)}{dx^4}|} = \max_x{|cos(x)|} = 1, for \ x \in [0, \pi]
    \]
    \\    
    To get the max of product, we differentiate it with respect to $x$ as follows:
    \[
      \frac{d}{dx}(\prod_{i=0}^{2} (x - x_i)) = 0,
    \]
    \[
      \implies x = \frac{(x_0 + x_1 + x_2)}{3}
    \]
    Substituting $x$ in (2):
    \[
      10^{-6} \leq \frac{1}{2!}\max|\frac{(x_1 - x_0) + (x_2 - x_0)}{3}\frac{(x_0 - x_1) + (x_2 - x_1)}{3}\frac{(x_0 - x_2) + (x_1 - x_2)}{3}|
    \]
    Let $h = x_1 - x_0 = x_2 - x_1$ be the table spacing,
    \[
      10^{-6} \leq \frac{1}{3!}|\frac{2h^3}{3}|
    \]
    \[
      \implies h \geq \sqrt[3]{9\times10^{-6}} = 0.0208
    \]
    Hence, the required table spacing is $h = 0.0208$.
    
    \item To get the number of entries:
    \begin{enumerate}
      \item For Linear Interpolation:
      \[
        N = \frac{\pi - 0}{h} = \frac{\pi}{0.0028} \approx 1111
      \]
      \item For Quadratic Interpolation:
      \[
        N = \frac{\pi - 0}{h} = \frac{\pi}{0.0208} \approx 151
      \]      
    \end{enumerate}
  \end{enumerate}
\end{enumerate}
\end{document}















